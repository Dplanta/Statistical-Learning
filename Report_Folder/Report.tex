% Options for packages loaded elsewhere
\PassOptionsToPackage{unicode}{hyperref}
\PassOptionsToPackage{hyphens}{url}
%
\documentclass[
]{article}
\usepackage{amsmath,amssymb}
\usepackage{lmodern}
\usepackage{iftex}
\ifPDFTeX
  \usepackage[T1]{fontenc}
  \usepackage[utf8]{inputenc}
  \usepackage{textcomp} % provide euro and other symbols
\else % if luatex or xetex
  \usepackage{unicode-math}
  \defaultfontfeatures{Scale=MatchLowercase}
  \defaultfontfeatures[\rmfamily]{Ligatures=TeX,Scale=1}
\fi
% Use upquote if available, for straight quotes in verbatim environments
\IfFileExists{upquote.sty}{\usepackage{upquote}}{}
\IfFileExists{microtype.sty}{% use microtype if available
  \usepackage[]{microtype}
  \UseMicrotypeSet[protrusion]{basicmath} % disable protrusion for tt fonts
}{}
\makeatletter
\@ifundefined{KOMAClassName}{% if non-KOMA class
  \IfFileExists{parskip.sty}{%
    \usepackage{parskip}
  }{% else
    \setlength{\parindent}{0pt}
    \setlength{\parskip}{6pt plus 2pt minus 1pt}}
}{% if KOMA class
  \KOMAoptions{parskip=half}}
\makeatother
\usepackage{xcolor}
\usepackage[margin=1in]{geometry}
\usepackage{color}
\usepackage{fancyvrb}
\newcommand{\VerbBar}{|}
\newcommand{\VERB}{\Verb[commandchars=\\\{\}]}
\DefineVerbatimEnvironment{Highlighting}{Verbatim}{commandchars=\\\{\}}
% Add ',fontsize=\small' for more characters per line
\usepackage{framed}
\definecolor{shadecolor}{RGB}{248,248,248}
\newenvironment{Shaded}{\begin{snugshade}}{\end{snugshade}}
\newcommand{\AlertTok}[1]{\textcolor[rgb]{0.94,0.16,0.16}{#1}}
\newcommand{\AnnotationTok}[1]{\textcolor[rgb]{0.56,0.35,0.01}{\textbf{\textit{#1}}}}
\newcommand{\AttributeTok}[1]{\textcolor[rgb]{0.77,0.63,0.00}{#1}}
\newcommand{\BaseNTok}[1]{\textcolor[rgb]{0.00,0.00,0.81}{#1}}
\newcommand{\BuiltInTok}[1]{#1}
\newcommand{\CharTok}[1]{\textcolor[rgb]{0.31,0.60,0.02}{#1}}
\newcommand{\CommentTok}[1]{\textcolor[rgb]{0.56,0.35,0.01}{\textit{#1}}}
\newcommand{\CommentVarTok}[1]{\textcolor[rgb]{0.56,0.35,0.01}{\textbf{\textit{#1}}}}
\newcommand{\ConstantTok}[1]{\textcolor[rgb]{0.00,0.00,0.00}{#1}}
\newcommand{\ControlFlowTok}[1]{\textcolor[rgb]{0.13,0.29,0.53}{\textbf{#1}}}
\newcommand{\DataTypeTok}[1]{\textcolor[rgb]{0.13,0.29,0.53}{#1}}
\newcommand{\DecValTok}[1]{\textcolor[rgb]{0.00,0.00,0.81}{#1}}
\newcommand{\DocumentationTok}[1]{\textcolor[rgb]{0.56,0.35,0.01}{\textbf{\textit{#1}}}}
\newcommand{\ErrorTok}[1]{\textcolor[rgb]{0.64,0.00,0.00}{\textbf{#1}}}
\newcommand{\ExtensionTok}[1]{#1}
\newcommand{\FloatTok}[1]{\textcolor[rgb]{0.00,0.00,0.81}{#1}}
\newcommand{\FunctionTok}[1]{\textcolor[rgb]{0.00,0.00,0.00}{#1}}
\newcommand{\ImportTok}[1]{#1}
\newcommand{\InformationTok}[1]{\textcolor[rgb]{0.56,0.35,0.01}{\textbf{\textit{#1}}}}
\newcommand{\KeywordTok}[1]{\textcolor[rgb]{0.13,0.29,0.53}{\textbf{#1}}}
\newcommand{\NormalTok}[1]{#1}
\newcommand{\OperatorTok}[1]{\textcolor[rgb]{0.81,0.36,0.00}{\textbf{#1}}}
\newcommand{\OtherTok}[1]{\textcolor[rgb]{0.56,0.35,0.01}{#1}}
\newcommand{\PreprocessorTok}[1]{\textcolor[rgb]{0.56,0.35,0.01}{\textit{#1}}}
\newcommand{\RegionMarkerTok}[1]{#1}
\newcommand{\SpecialCharTok}[1]{\textcolor[rgb]{0.00,0.00,0.00}{#1}}
\newcommand{\SpecialStringTok}[1]{\textcolor[rgb]{0.31,0.60,0.02}{#1}}
\newcommand{\StringTok}[1]{\textcolor[rgb]{0.31,0.60,0.02}{#1}}
\newcommand{\VariableTok}[1]{\textcolor[rgb]{0.00,0.00,0.00}{#1}}
\newcommand{\VerbatimStringTok}[1]{\textcolor[rgb]{0.31,0.60,0.02}{#1}}
\newcommand{\WarningTok}[1]{\textcolor[rgb]{0.56,0.35,0.01}{\textbf{\textit{#1}}}}
\usepackage{longtable,booktabs,array}
\usepackage{calc} % for calculating minipage widths
% Correct order of tables after \paragraph or \subparagraph
\usepackage{etoolbox}
\makeatletter
\patchcmd\longtable{\par}{\if@noskipsec\mbox{}\fi\par}{}{}
\makeatother
% Allow footnotes in longtable head/foot
\IfFileExists{footnotehyper.sty}{\usepackage{footnotehyper}}{\usepackage{footnote}}
\makesavenoteenv{longtable}
\usepackage{graphicx}
\makeatletter
\def\maxwidth{\ifdim\Gin@nat@width>\linewidth\linewidth\else\Gin@nat@width\fi}
\def\maxheight{\ifdim\Gin@nat@height>\textheight\textheight\else\Gin@nat@height\fi}
\makeatother
% Scale images if necessary, so that they will not overflow the page
% margins by default, and it is still possible to overwrite the defaults
% using explicit options in \includegraphics[width, height, ...]{}
\setkeys{Gin}{width=\maxwidth,height=\maxheight,keepaspectratio}
% Set default figure placement to htbp
\makeatletter
\def\fps@figure{htbp}
\makeatother
\setlength{\emergencystretch}{3em} % prevent overfull lines
\providecommand{\tightlist}{%
  \setlength{\itemsep}{0pt}\setlength{\parskip}{0pt}}
\setcounter{secnumdepth}{-\maxdimen} % remove section numbering
\ifLuaTeX
  \usepackage{selnolig}  % disable illegal ligatures
\fi
\IfFileExists{bookmark.sty}{\usepackage{bookmark}}{\usepackage{hyperref}}
\IfFileExists{xurl.sty}{\usepackage{xurl}}{} % add URL line breaks if available
\urlstyle{same} % disable monospaced font for URLs
\hypersetup{
  pdftitle={Titolo del progetto},
  pdfauthor={Domenico Plantamura, Eduardo David Lotto, Manuel D'Alterio Grazioli, Gabriele Fugagnoli},
  hidelinks,
  pdfcreator={LaTeX via pandoc}}

\title{Titolo del progetto}
\author{Domenico Plantamura, Eduardo David Lotto, Manuel D'Alterio
Grazioli, Gabriele Fugagnoli}
\date{}

\begin{document}
\maketitle

{
\setcounter{tocdepth}{2}
\tableofcontents
}
\hfill\break

\hypertarget{introduction}{%
\section{Introduction}\label{introduction}}

\hypertarget{objective-of-the-project}{%
\subsection{Objective of the project}\label{objective-of-the-project}}

Our goal is to investigate whether the salaries earned by the NBA
players during the 2023-2024 season are fair in proportion to their
performance during the current year's Regular season. To analyse
performance, we selected several statistics: from the most common such
as points, rebounds, assists to advanced metrics like Usage, Player
Impact Estimated and Winning Shares. The idea is to deep dive into the
relationship between salaries and performance through different models
in order to understand what kind of relationship there is and which
model best fits the data. Finally, we will compare actual salaries with
those predicted by our models to find out which players (according to
the models) are the most overpaid or underpaid.

\emph{AGGIUNGERE DETTAGLI E SPIEGARE IN SINTESI IL WEB SCRAPING
REALIZZATO}

\hypertarget{steps-followed}{%
\subsection{Steps followed}\label{steps-followed}}

To perform our analysis we followed these steps:

\begin{enumerate}
\def\labelenumi{\arabic{enumi}.}
\item
  Data collection;
\item
  Data exploration;
\item
  Analysis;
\item
  Interpretation.
\end{enumerate}

We now explain in depth each step.

\hfill\break

\hypertarget{data-collection}{%
\section{Data collection}\label{data-collection}}

We performed a web scraping operation from the
\href{https://www.nba.com/stats}{Official NBA Stats} website, from which
we collected most of the stats. Additionally, we downloaded data about
the salaries from \url{https://hoopshype.com/} and other stats of
interest from \url{https://www.basketball-reference.com}. All data
concerns the 2023-2024 NBA Regular Season.

\hypertarget{glossary}{%
\subsection{Glossary}\label{glossary}}

\begin{itemize}
\tightlist
\item
  PLAYER NAME: players' name
\item
  SALARY: salary earned by a player for 2023-2024 season (collected from
  hoopshype)
\item
  AGE: players' age
\item
  POS: ``Position'', states the playing position of a player
\end{itemize}

TRADITIONAL STATS (collected from NBA.com)

\begin{itemize}
\tightlist
\item
  GP: ``Games played'', the number of games played by a player during
  the 2023-2024 regular season
\item
  FG\_PCT: ``Field Goal Percentage'', The percentage of field goal
  attempts that a player makes. Formula: (FGM)/(FGA)
\item
  FG3\_PCT: ``3 Points ``Field Goal Percentage'', The percentage of 3pt
  field goal attempts that a player makes.
\item
  FT\_PCT: ``Free throws Percentage'', the percentage of free throws
  attempts that a player makes
\item
  OREB: ``Offensive Rebounds'', The number of rebounds a player or team
  has collected while they were on offense
\item
  DREB: ``Defensive Rebounds'', The number of rebounds a player or team
  has collected while they were on defense
\item
  REB: ``Rebounds''; A rebound occurs when a player recovers the ball
  after a missed shot. This statistic is the number of total rebounds a
  player or team has collected on either offense or defense
\item
  AST: ``Assists'', The number of assists -- passes that lead directly
  to a made basket -- by a player
\item
  TOV: ``Turnovers''; A turnover occurs when the player or team on
  offense loses the ball to the defense
\item
  STL: ``Steals'', Number of times a defensive player or team takes the
  ball from a player on offense, causing a turnover
\item
  BLK: ``Blocks'', A block occurs when an offensive player attempts a
  shot, and the defense player tips the ball, blocking their chance to
  score
\item
  BLKA: ``Blocks Against'', The number of shots attempted by a player or
  team that are blocked by a defender
\item
  PF: ``Personal fouls'', The number of personal fouls a player or team
  committed
\item
  PFD: ``Personal fouls drawn'', The number of personal fouls that are
  drawn by a player or team
\item
  PTS: ``Points'', the number of points scored by a player
\item
  MIN: ``Minutes played'', number of minutes played by a player during
  the 2023-2024 Regular season
\item
  MIN\_G: ``Minutes played per game''
\end{itemize}

ADVANCED STATS (collected from NBA.com)

\begin{itemize}
\tightlist
\item
  OFF\_RATING: ``Offensive Rating'', Measures a team's points scored per
  100 possessions. On a player level this statistic is team points
  scored per 100 possessions while they are on court. Formula:
  100*((Points)/(POSS)
\item
  DEF\_RATING: ``Defensive Rating'', The number of points allowed per
  100 possessions by a team. For a player, it is the number of points
  per 100 possessions that the team allows while that individual player
  is on the court. Formula: 100*((Opp Points)/(Opp POSS))
\item
  NET\_RATING: ``Net Rating'', Measures a team's point differential per
  100 possessions. On player level this statistic is the team's point
  differential per 100 possessions while they are on court. Formula:
  OFFRTG - DEFRTG
\item
  AST\_TO: ``Assist to Turnover Ratio'', The number of assists for a
  player or team compared to the number of turnovers they have committed
\item
  TS\_PCT: ``True Shooting Percentage'', A shooting percentage that
  factors in the value of three-point field goals and free throws in
  addition to conventional two-point field goals. Formula: Points/
  {[}2\emph{(Field Goals Attempted+0.44}Free Throws Attempted){]}
\item
  USG\_PCT: ``Usage Percentage'', The percentage of team plays used by a
  player when they are on the floor. Formula: (FGA + Possession Ending
  FTA + TO) / POSS
\item
  PIE: ``Player Impact Estimate'', PIE measures a player's overall
  statistical contribution against the total statistics in games they
  play in. PIE yields results which are comparable to other advanced
  statistics (e.g.~PER) using a simple formula. Formula: (PTS + FGM +
  FTM - FGA - FTA + DREB + (.5 * OREB) + AST + STL + (.5 * BLK) - PF -
  TO) / (GmPTS + GmFGM + GmFTM - GmFGA - GmFTA + GmDREB + (.5 * GmOREB)
  + GmAST + GmSTL + (.5 * GmBLK) - GmPF - GmTO)
\end{itemize}

The stats below are collected from Basketball reference:

\begin{itemize}
\tightlist
\item
  WS: ``Win Shares''; an estimate of the number of wins contributed by a
  player.
\item
  BPM: ``Box Plus/Minus''; a box score estimate of the points per 100
  possessions that a player contributed above a league-average player,
  translated to an average team
\item
  VORP: ``Value Over Replacement Player''; a box score estimate of the
  points per 100 TEAM possessions that a player contributed above a
  replacement-level (-2.0) player, translated to an average team and
  prorated to an 82-game season. Multiply by 2.70 to convert to wins
  over replacement.
\end{itemize}

BPM and VORP are calculated per 100 possessions; MIN is a total stat for
the whole regular season, MIN\_G is calculated per game. The other stats
are considered per 48 minutes. (vedi WS)

\emph{AGGIUNGERE DETTAGLI, LE ROBE DEI 48 MINUTI ETC ETC, qui c'è
davvero molto da scrivere da quanto ho capito ed è uno dei punti forti
del progetto}

\hypertarget{data-integration}{%
\subsection{Data integration}\label{data-integration}}

\emph{Spiegare tutta la roba di come si sono unite le tabelle,
aggiustato i dati come i nomi etc etc}

\begin{Shaded}
\begin{Highlighting}[]
\NormalTok{data\_st }\OtherTok{\textless{}{-}} \FunctionTok{merge}\NormalTok{(data\_salary, data\_traditional\_per48, }\AttributeTok{by =} \StringTok{"PLAYER\_NAME"}\NormalTok{, }\AttributeTok{all =} \ConstantTok{TRUE}\NormalTok{)}
\NormalTok{data\_ast }\OtherTok{\textless{}{-}} \FunctionTok{merge}\NormalTok{(data\_st, data\_advanced, }\AttributeTok{by =} \StringTok{"PLAYER\_NAME"}\NormalTok{, }\AttributeTok{all =} \ConstantTok{TRUE}\NormalTok{)}
\NormalTok{data\_mast }\OtherTok{\textless{}{-}} \FunctionTok{merge}\NormalTok{(data\_ast, data\_miscellaneous, }\AttributeTok{by =} \StringTok{"PLAYER\_NAME"}\NormalTok{, }\AttributeTok{all =} \ConstantTok{TRUE}\NormalTok{)}
\NormalTok{data\_mastt }\OtherTok{\textless{}{-}} \FunctionTok{merge}\NormalTok{(data\_mast, data\_trad\_tot, }\AttributeTok{by =} \StringTok{"PLAYER\_NAME"}\NormalTok{, }\AttributeTok{all =} \ConstantTok{TRUE}\NormalTok{)}
\NormalTok{final\_dataset }\OtherTok{\textless{}{-}} \FunctionTok{merge}\NormalTok{(data\_mastt, data\_vorp, }\AttributeTok{by =} \StringTok{"PLAYER\_NAME"}\NormalTok{, }\AttributeTok{all =} \ConstantTok{TRUE}\NormalTok{)}
\end{Highlighting}
\end{Shaded}

\hypertarget{data-exploration}{%
\subsection{Data exploration}\label{data-exploration}}

\begin{longtable}[]{@{}lllcccccccc@{}}
\toprule()
& PLAYER\_NAME & Salary & AGE & GP & FG\_PCT & FG3\_PCT & FT\_PCT & OREB
& DREB & REB \\
\midrule()
\endhead
3 & Aaron Gordon & 22266182 & 28 & 73 & 0.556 & 0.290 & 0.658 & 3.6 &
6.2 & 9.8 \\
4 & Aaron Holiday & 2346614 & 27 & 78 & 0.446 & 0.387 & 0.921 & 0.9 &
3.8 & 4.7 \\
5 & Aaron Nesmith & 5634257 & 24 & 72 & 0.496 & 0.419 & 0.781 & 1.5 &
5.1 & 6.6 \\
6 & Aaron Wiggins & 1836096 & 25 & 78 & 0.562 & 0.492 & 0.789 & 2.3 &
4.9 & 7.3 \\
12 & Al Horford & 10000000 & 37 & 65 & 0.511 & 0.419 & 0.867 & 2.3 & 9.1
& 11.4 \\
\bottomrule()
\end{longtable}

\begin{longtable}[]{@{}lcccccccccc@{}}
\toprule()
& AST & TOV & STL & BLK & BLKA & PF & PTS & OFF\_RATING & DEF\_RATING &
NET\_RATING \\
\midrule()
\endhead
3 & 5.4 & 2.2 & 1.2 & 0.9 & 1.2 & 3.0 & 21.2 & 119.8 & 111.1 & 8.7 \\
4 & 5.3 & 2.0 & 1.6 & 0.2 & 0.8 & 4.7 & 19.4 & 110.5 & 107.6 & 2.9 \\
5 & 2.6 & 1.5 & 1.6 & 1.2 & 1.2 & 5.8 & 21.1 & 119.3 & 115.0 & 4.3 \\
6 & 3.4 & 2.2 & 2.2 & 0.7 & 1.3 & 3.6 & 21.2 & 115.6 & 110.0 & 5.7 \\
12 & 4.6 & 1.3 & 1.0 & 1.7 & 0.3 & 2.6 & 15.5 & 120.9 & 109.5 & 11.4 \\
\bottomrule()
\end{longtable}

\begin{longtable}[]{@{}lccccccccccc@{}}
\toprule()
& AST\_TO & TS\_PCT & USG\_PCT & PIE & PFD & MIN & MIN\_G & Pos & WS &
BPM & VORP \\
\midrule()
\endhead
3 & 2.47 & 0.607 & 0.174 & 0.103 & 4.7 & 2296.810 & 31.46315 & PF & 7.1
& 1.3 & 1.9 \\
4 & 2.64 & 0.578 & 0.158 & 0.078 & 2.5 & 1269.297 & 16.27303 & PG & 2.5
& -1.5 & 0.2 \\
5 & 1.69 & 0.631 & 0.158 & 0.071 & 3.5 & 1994.655 & 27.70354 & SF & 4.1
& -0.5 & 0.8 \\
6 & 1.54 & 0.664 & 0.163 & 0.096 & 2.3 & 1227.938 & 15.74280 & SG & 3.7
& 0.7 & 0.8 \\
12 & 3.50 & 0.650 & 0.119 & 0.105 & 0.8 & 1739.797 & 26.76610 & C & 6.2
& 3.6 & 2.5 \\
\bottomrule()
\end{longtable}

\end{document}
